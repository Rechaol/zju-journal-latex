\documentclass[UTF8, 12pt]{ctexart}
\usepackage{graphicx}
\usepackage{amsmath}
\usepackage{booktabs}
\usepackage{float}
\usepackage{caption}
\usepackage{geometry}
\geometry{a4paper, margin=2.5cm}

% 图表标题双语设置(与文档1一致)
\captionsetup[figure]{name={图}, labelsep=quad, format=hang, font=small, labelfont=bf}
\captionsetup[table]{name={表}, labelsep=quad, format=hang, font=small, labelfont=bf}
\renewcommand{\figurename}{图}
\renewcommand{\tablename}{表}



\begin{document}

% 中文标题与作者(对应文档1结构)
\title{中文标题}

\author{张三丰(XX学院XX专业班级,学号)}

\date{}
\maketitle

% 中文摘要(包含目的、方法、结果、结论,第三人称)
\begin{abstract}

尽量写成报道性摘要,包括目的、方法、结果和结论四部分(200~300字)。摘要应具独立性和自含性,采用第三人称的写法,不必以"本文""作者"等做主语。

\end{abstract}

\noindent

\textbf{关键词:} 关键词一;关键词二;关键词三;关键词三(关键词选词要规范、术语化、具体、能直观反映本文要点)

% 英文标题、作者、摘要(与中文一致,第三人称)
\renewcommand{\abstractname}{Title}

\begin{center}
  \textbf{Title}\\
  \vspace{0.5em}
  Zhang Sanfeng (XX College XX Major Class, Student ID)
\end{center}

\begin{abstract}

    English abstract should be consistent with Chinese abstract in meaning, and the length can be appropriately extended if necessary. It should also be written in the third person.

\end{abstract}

\noindent
\textbf{Key Words:} Boolean function; tabular methods; minterm table; e derivative; cryptography (中英文关键词须一一对应)

% 正文结构(与文档1一致,引言可选)
\section{引言}

引言可要可不要,根据具体情况定。……

\section{第1节标题}

第1节的正文。……

% 公式示例(居中编号)
定义1 设 \( f(x) \) 为……
\[ f(x) = x^2 + 2x + 1 \]
式(1)中,\( x \) 表示XXXX,由式可知,……。

% 三线表示例(中英文标题对照)
\begin{table}[H]
  \centering
  \caption{中文名称(三线表)}
  \label{tab:table1}
  \begin{tabular}{ccc}
    \toprule
    列1 & 列2 & 列3 \\
    \midrule
    内容1 & 内容2 & 内容3 \\
    \bottomrule
  \end{tabular}
  \small{\textit{Table 1. The table’s English title}}
\end{table}

% 参考文献格式(与文档1一致,包含中英文对照)
\section{参考文献}
张三丰,李四. XX理论研究[M]. 杭州:浙江大学出版社,2001. \\
ZHANG S F, LI S. A study of XXX[M]. Hangzhou: Zhejiang University Press, 2001. \\
赵六. XX理论的方法研究[D]. 杭州:浙江大学,2007. \\
ZHAO L. Research on XXX[D]. Hangzhou: Zhejiang University, 2007. \\
张三,李四,王五,等. 对XX的研究[J]. 浙江大学学报(理学版),2013,40(1):1-4. \\
ZHANG S, LI S, WANG W, et al. The research of XX[J]. Journal of Zhejiang University (Science Edition), 2023, 40(1): 1-4. \\
LI W W, WANG Z. The e-derivative of Boolean functions and its application in the fault detection and cryptographic system[J]. Kybernetes, 2008, 37(2): 49-65. \\
李卫卫,王卓,张志杰. 导数和e-导数在研究H布尔函数中的应用[C]//中国通信学会第五届学术会议论文集. 北京:中国通信学会, 2008:267-271. \\
LI W W, WANG Z, ZHANG Z J. The application of derivative and e-derivative on H-Boolean function[C]//CHINA SCI-TEC. Beijing: China Institute of Communications, 2008, 1(1): 267-271.
\end{document}